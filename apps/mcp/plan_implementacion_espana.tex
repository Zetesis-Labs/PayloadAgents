\documentclass[12pt,a4paper]{article}

% Paquetes
\usepackage[utf8]{inputenc}
\usepackage[spanish]{babel}
\usepackage[margin=2.5cm]{geometry}
\usepackage{setspace}
\usepackage{parskip}
\usepackage{titlesec}
\usepackage{fancyhdr}
\usepackage{xcolor}
\usepackage{hyperref}
\usepackage{csquotes}
\usepackage{enumitem}
\usepackage{array}
\usepackage{longtable}
\usepackage{booktabs}

% Configuración de hipervínculos
\hypersetup{
    colorlinks=true,
    linkcolor=blue!70!black,
    urlcolor=blue!70!black,
    citecolor=blue!70!black
}

% Configuración de encabezados
\pagestyle{fancy}
\fancyhf{}
\rhead{Plan de Implementación}
\lhead{Justicia Sin Estado - España}
\cfoot{\thepage}

% Formato de secciones
\titleformat{\section}{\Large\bfseries}{\thesection.}{0.5em}{}
\titleformat{\subsection}{\large\bfseries}{\thesubsection.}{0.5em}{}
\titleformat{\subsubsection}{\normalsize\bfseries}{\thesubsubsection.}{0.5em}{}

% Espaciado
\onehalfspacing

% Comando para citas textuales
\newcommand{\cita}[1]{\begin{quote}\textit{``#1''}\end{quote}}

% Colores para fases
\definecolor{fase1}{RGB}{46, 125, 50}
\definecolor{fase2}{RGB}{255, 152, 0}
\definecolor{fase3}{RGB}{211, 47, 47}
\definecolor{fase4}{RGB}{156, 39, 176}

\begin{document}

% Portada
\begin{titlepage}
    \centering
    \vspace*{2cm}

    {\Huge\bfseries Plan de Implementación Gradual}\\[0.5cm]
    {\LARGE\bfseries Hacia una Justicia Sin Estado en España}\\[1cm]
    {\Large Basado en el Pensamiento de Miguel Anxo Bastos}\\[2cm]

    \vfill

    \begin{center}
    \fbox{\parbox{0.8\textwidth}{
        \centering
        \textbf{Principio Rector}\\[0.3cm]
        \textit{``Las transiciones históricas aparecían las nuevas tecnologías y eran adoptadas sólo si presentaban ventajas frente a las viejas. Nunca de forma imperativa o forzada. La gente las prefería porque eran mejores.''}\\[0.2cm]
        --- Miguel Anxo Bastos
    }}
    \end{center}

    \vfill

    {\large Febrero 2026}
\end{titlepage}

% Índice
\tableofcontents
\newpage

% ============================================================================
\section{Introducción y Principios Metodológicos}
% ============================================================================

\subsection{El Rechazo a la Planificación Centralizada}

Antes de presentar cualquier plan de transición, es fundamental establecer que este documento no pretende ser un programa de ingeniería social impuesto desde arriba. Siguiendo el pensamiento de Bastos, las transiciones exitosas son aquellas que emergen orgánicamente:

\cita{Tú mira, en cambio, cómo llegamos al teléfono móvil. Fue haciéndose poco a poco. No se puso un plazo, no se obligó a todo el mundo a cambiar el teléfono viejo por uno nuevo, sino que nos fuimos adaptando, se fueron creando una serie de antenas y se hizo una transición. Pero no hubo una fecha ni una planificación.}

Por tanto, este plan debe entenderse como un \textbf{marco orientativo} que identifica oportunidades y vectores de cambio, no como un programa con plazos imperativos.

\subsection{Principios de Transición}

\begin{enumerate}
    \item \textbf{Gradualismo orgánico}: Los cambios deben ser adoptados porque son mejores, no impuestos por decreto.
    \item \textbf{Descentralización}: Cada medida debe fragmentar el poder, nunca concentrarlo.
    \item \textbf{Irreversibilidad institucional}: Crear arquitecturas que dificulten la regresión al estatismo.
    \item \textbf{Competencia institucional}: Permitir que diferentes jurisdicciones experimenten y compitan.
    \item \textbf{Respeto a la ley natural}: Todo cambio debe fundamentarse en principios de no agresión y propiedad.
\end{enumerate}

\cita{Las instituciones económicas y políticas que conducen al desarrollo económico y social no son fácilmente exportables y mucho menos por la fuerza. La imitación y la adaptación pacífica de normas acostumbra a ser más eficaz.}

% ============================================================================
\section{Fase I: Aprovechamiento del Marco Existente}
% ============================================================================

{\color{fase1}\rule{\linewidth}{2pt}}

Esta fase no requiere cambios legislativos profundos. Se trata de utilizar las estructuras ya existentes para maximizar la libertad dentro del sistema actual.

\subsection{Competencia Fiscal Autonómica}

España ya posee un sistema autonómico que permite cierta competencia institucional. Esta es una ventaja a explotar:

\cita{Los impuestos en España, si no hubiera un sistema autonómico, serían más altos. Por ejemplo, seguiríamos todos con el impuesto de sucesiones. Pero basta con que una comunidad lo quite para que las demás vecinas empiecen a bajarlo también, por pura competencia fiscal.}

\subsubsection{Medidas Concretas}

\begin{itemize}
    \item \textbf{Eliminación progresiva del Impuesto de Sucesiones} en comunidades autónomas receptivas, generando efecto contagio.
    \item \textbf{Reducción del tramo autonómico del IRPF} en comunidades con superávit.
    \item \textbf{Flexibilización de normativas laborales} donde las competencias lo permitan.
    \item \textbf{Simplificación de licencias de actividad} a nivel autonómico y municipal.
\end{itemize}

\subsection{Extensión del Modelo Foral}

Los fueros vasco y navarro constituyen un precedente valioso:

\cita{Fueros quedan en el País Vasco y en Navarra, restos de aquellos fueros viejos. Y son fueros de verdad porque ni siquiera es el País Vasco quien recauda tributos, sino que son las haciendas forales de los viejos territorios. [...] Si es un privilegio y es tan bueno, extendámoslo a toda España.}

\subsubsection{Medidas Concretas}

\begin{itemize}
    \item Promover el debate sobre la \textbf{generalización del concierto económico}.
    \item Apoyar iniciativas de \textbf{soberanía fiscal autonómica}.
    \item Defender la \textbf{asunción de deuda propia} por cada comunidad.
\end{itemize}

\subsection{Arbitraje y Resolución Alternativa de Conflictos}

Promover el uso de mecanismos ya legales de justicia privada:

\begin{itemize}
    \item Expansión de \textbf{cláusulas de arbitraje} en contratos comerciales y civiles.
    \item Fomento de \textbf{tribunales arbitrales sectoriales} (comercio, construcción, tecnología).
    \item Desarrollo de \textbf{mediación comunitaria} a nivel de barrio y municipio.
    \item Promoción de \textbf{seguros de defensa jurídica} privados.
\end{itemize}

% ============================================================================
\section{Fase II: Descentralización y Fragmentación}
% ============================================================================

{\color{fase2}\rule{\linewidth}{2pt}}

Esta fase requiere reformas legislativas pero dentro del marco constitucional existente.

\subsection{Transferencia de Competencias a Municipios}

\cita{La descentralización no sólo frena el poder político, sino que a pequeña escala las funciones sociales se ejercen mejor.}

\subsubsection{Servicios Municipalizables}

\begin{itemize}
    \item \textbf{Seguridad local}: Policías municipales con mayor autonomía operativa.
    \item \textbf{Urbanismo}: Eliminación de normativas estatales uniformes.
    \item \textbf{Servicios sociales}: Gestión vecinal de la ayuda a necesitados.
    \item \textbf{Educación básica}: Currículos adaptados a realidades locales.
\end{itemize}

\cita{Determinadas normas es mejor que se puedan frenar, que se puedan modular o que se puedan adaptar al terreno. Muchas leyes que se hacen no tienen en cuenta las peculiaridades de cada sitio. Hacen una ley para toda España, sea para Madrid, sea para un pescador de Galicia que no tiene horarios de trabajo de 8:00 a 15:00.}

\subsection{Comunidades Vecinales Autogestionadas}

Recuperar formas tradicionales de organización comunitaria:

\cita{¿Y si le digo yo que los vecinos de mi zona se organizaron y asfaltamos el camino nosotros a escote? Le puede parecer súper extraño, pero así se hizo. ¿O si le digo que muchos de nosotros tenemos pozos de agua privados o que incluso puede haber generadores eléctricos a una escala mucho más pequeña que la actual?}

\subsubsection{Modelos a Promover}

\begin{itemize}
    \item \textbf{Comunidades de montes vecinales}: Ya existentes en Galicia, extender el modelo.
    \item \textbf{Urbanizaciones privadas}: Con servicios propios (seguridad, mantenimiento, gestión).
    \item \textbf{Cooperativas de servicios}: Electricidad, agua, telecomunicaciones a pequeña escala.
    \item \textbf{Asociaciones de comerciantes}: Gestión privada de espacios comerciales y seguridad.
\end{itemize}

\cita{No es difícil, si tenemos un poco de imaginación histórica, comunidades al estilo de la Edad Media [...] de pequeña extensión geográfica pero compartiendo valores, moneda y derecho. Sólo que en el futuro estarían desprovistas de formas violentas de ejercicio del poder.}

\subsection{Desregulación Sectorial}

\cita{La sanidad y la educación están encarecidas adrede: patentes, licencias médicas, cupo restringido de médicos... Lo que encarece son estas regulaciones.}

\subsubsection{Áreas Prioritarias}

\begin{itemize}
    \item \textbf{Vivienda}: Eliminar restricciones urbanísticas que encarecen artificialmente.
    \item \textbf{Sanidad}: Permitir más facultades de medicina, reconocer títulos extranjeros.
    \item \textbf{Educación}: Liberalizar homologaciones, permitir modelos alternativos.
    \item \textbf{Comercio}: Eliminar licencias de apertura y horarios obligatorios.
    \item \textbf{Transporte}: Desregular taxi, VTC, transporte de mercancías.
\end{itemize}

\cita{La vivienda muchas veces es encarecida artificialmente con regulaciones, con licencias...}

% ============================================================================
\section{Fase III: Privatización Real de Servicios}
% ============================================================================

{\color{fase3}\rule{\linewidth}{2pt}}

Bastos distingue entre externalización (que sigue siendo pública) y privatización real:

\cita{Privatización estrictamente es el abandono, que el estado no preste un servicio. [...] Lo que nosotros llamamos escuela concertada no es una escuela privada, es una escuela pública, sólo que de gestión privada. Porque, al final, el currículo, la financiación y todo lo marca el estado.}

\subsection{Educación}

\subsubsection{Transición}

\begin{enumerate}
    \item \textbf{Cheque escolar universal}: Los padres eligen centro, el dinero sigue al alumno.
    \item \textbf{Liberalización curricular}: Centros definen sus propios programas.
    \item \textbf{Eliminación de titulaciones obligatorias}: El mercado valora competencias.
    \item \textbf{Desaparición gradual del cheque}: Las familias pagan directamente.
\end{enumerate}

\subsection{Sanidad}

\cita{Tan pública es la prestación directa como la externalizada, pues lo que determina la característica de ser o no público es el financiamiento, no la prestación. [...] Es tan público el NHS británico como el sistema sanitario alemán o suizo, que aquí serían denominados privatizados a pesar de constituir buenos ejemplos de una excelente sanidad pública.}

\subsubsection{Transición}

\begin{enumerate}
    \item \textbf{Modelo suizo/alemán}: Seguro obligatorio privado con competencia entre aseguradoras.
    \item \textbf{Cuentas de ahorro sanitario}: Deducciones fiscales para ahorro sanitario personal.
    \item \textbf{Eliminación del seguro obligatorio}: Cada persona decide su nivel de cobertura.
    \item \textbf{Caridad privada}: Organizaciones religiosas y civiles atienden a indigentes.
\end{enumerate}

\cita{Yo soy contrario al Estado de bienestar. Yo creo en soluciones privadas: pensiones privadas, educación privada, sanidad privada y ayuda a los pobres de forma privada, que funciona muy bien.}

\subsection{Pensiones}

\subsubsection{Transición}

\begin{enumerate}
    \item \textbf{Sistema de capitalización paralelo}: Opción voluntaria de salirse del sistema público.
    \item \textbf{Reducción progresiva de cotizaciones}: A medida que el sistema público se contrae.
    \item \textbf{Fondos de pensiones privados}: Con libertad de inversión.
    \item \textbf{Responsabilidad familiar}: Recuperación del modelo tradicional de cuidado de mayores.
\end{enumerate}

\subsection{Seguridad}

\cita{No tiene por qué darse un monopolio territorial de la defensa, de la misma forma que nuestros vecinos no tienen por qué contratar con la misma aseguradora de incendios.}

\subsubsection{Transición}

\begin{enumerate}
    \item \textbf{Expansión de seguridad privada}: Liberalizar requisitos y ámbitos de actuación.
    \item \textbf{Patrullas vecinales}: Legalizar y regular mínimamente.
    \item \textbf{Comunidades cerradas}: Con seguridad propia integral.
    \item \textbf{Derecho de autodefensa}: Liberalización de armas para defensa del hogar.
\end{enumerate}

\cita{Violenta es la gente, no las armas. Además, fíjese en una cosa: el Estado no le puede proteger a usted. El Estado, en el mejor de los casos, lo que puede hacer es detener a quien le agredió a usted, que es muy distinto.}

% ============================================================================
\section{Fase IV: Pluralismo Jurídico y Justicia Privada}
% ============================================================================

{\color{fase4}\rule{\linewidth}{2pt}}

Esta es la fase más ambiciosa, que requiere un cambio cultural profundo y posiblemente constitucional.

\subsection{Recuperación del Derecho Consuetudinario}

\cita{En la Edad Media, cada persona tenía una ley distinta según el fuero que tenía. Había en el mismo sitio varias leyes.}

\subsubsection{Medidas}

\begin{itemize}
    \item Reconocimiento legal de \textbf{costumbres locales} como fuente de derecho.
    \item Recuperación de \textbf{tribunales consuetudinarios} (como el Tribunal de las Aguas de Valencia).
    \item Promoción de \textbf{códigos de conducta sectoriales} con fuerza vinculante entre adherentes.
\end{itemize}

\subsection{Sistema de Justicia Competitivo}

\cita{Un juez estatal no puede ser neutral. Esta es la trampa. Un juez funcionario es lo menos neutral que puede haber.}

\subsubsection{Modelo Propuesto}

\begin{enumerate}
    \item \textbf{Tribunales privados acreditados}: Compiten por reputación y eficiencia.
    \item \textbf{Elección de jurisdicción}: Las partes acuerdan qué tribunal resuelve.
    \item \textbf{Ejecución mediante seguros}: Aseguradoras garantizan cumplimiento de sentencias.
    \item \textbf{Registro público de sentencias}: Transparencia para construir reputación judicial.
\end{enumerate}

\subsection{Modelo Lex Mercatoria}

\cita{En una sociedad de este tipo [ancap] la regulación sería más parecida a la Lex Mercatoria o a la regulación que rige en los mercados de bonos que a las actuales regulaciones financieras. Recordemos que el fragmentado mundo medieval operaba de una forma análoga y que, contrariamente a lo que se acostumbra a pensar, contaba con un corpus legal bastante homogéneo.}

\subsubsection{Características}

\begin{itemize}
    \item Derecho emergente de la práctica comercial, no impuesto por legislador.
    \item Arbitraje sectorial especializado.
    \item Sanciones reputacionales (exclusión de mercados, listas negras).
    \item Garantías mediante depósitos y seguros.
\end{itemize}

\subsection{Derecho de Secesión Municipal}

\cita{En Liechtenstein, un país de 60 mil habitantes, cada uno de sus municipios tiene derecho de secesión. Y se pueden marchar cuando quieran. ¿Y por qué incido en la secesión? Porque si la permanencia en un Estado o no es voluntaria, eso se aproxima mucho a la anarquía.}

\subsubsection{Propuesta}

\begin{itemize}
    \item Reconocimiento constitucional del \textbf{derecho de secesión municipal}.
    \item Procedimiento claro: referéndum local con mayoría cualificada.
    \item Municipios secesionados pueden \textbf{asociarse libremente} a otras jurisdicciones.
    \item Creación de \textbf{zonas francas} con mínima intervención estatal.
\end{itemize}

% ============================================================================
\section{Infraestructura Cultural e Intelectual}
% ============================================================================

Ninguna reforma institucional es sostenible sin un cambio cultural previo:

\cita{No basta simplemente con tener un sistema de mercado, es decir, hay que tener una sociedad que quiera ser capitalista. Necesita gente que sea frugal, necesita gente que sea trabajadora, necesita gente que sea seria y cumpla los contratos.}

\subsection{Educación en Valores}

\begin{itemize}
    \item Recuperación de la \textbf{ética del trabajo y el ahorro}.
    \item Enseñanza del \textbf{derecho natural} y principios de no agresión.
    \item Difusión de la \textbf{historia del derecho español} (fueros, concejos, gremios).
    \item Formación en \textbf{resolución privada de conflictos}.
\end{itemize}

\subsection{Papel de las Comunidades Religiosas}

\cita{Y sirven también para frenar el poder del estado, porque imponen una moral autónoma que no depende de la moral del estado. Y el estado normalmente quiere minar esa moral.}

\begin{itemize}
    \item Las instituciones religiosas como \textbf{contrapeso moral} al Estado.
    \item Caridad religiosa como \textbf{alternativa a servicios sociales estatales}.
    \item Educación confesional como \textbf{transmisora de valores}.
\end{itemize}

\subsection{Redes de Apoyo Mutuo}

\cita{Una hipotética sociedad sin Estado no estaría compuesta de individuos atomizados maximizadores de utilidad como plantean los economistas neoclásicos, sino de comunidades fuertemente cohesionadas de tal forma que se presten a la acción común cuando sea necesario.}

\begin{itemize}
    \item Fomento de \textbf{sociedades de socorros mutuos}.
    \item Creación de \textbf{cooperativas de servicios} (sanidad, educación, seguros).
    \item Fortalecimiento de \textbf{vínculos familiares extensos}.
    \item Desarrollo de \textbf{asociaciones profesionales} con funciones de protección.
\end{itemize}

% ============================================================================
\section{Garantías de Irreversibilidad}
% ============================================================================

\cita{Una de las cosas que se le critican a Milei es que no descentralizó. Y eso es muy importante. Porque es la forma de garantizar que, si hace cambios, no se vayan a revertir.}

\subsection{Principios de Diseño Institucional}

\begin{enumerate}
    \item \textbf{Fragmentación del poder}: Cada reforma debe dispersar competencias, nunca concentrarlas.
    \item \textbf{Competencia jurisdiccional}: Permitir que ciudadanos ``voten con los pies''.
    \item \textbf{Constitucionalización de límites}: Techos de gasto, prohibición de déficit, límites fiscales.
    \item \textbf{Privatización real}: No externalización que pueda revertirse, sino abandono de funciones.
    \item \textbf{Propiedad privada clara}: Derechos de propiedad bien definidos y protegidos.
\end{enumerate}

\cita{Es crear una arquitectura institucional que dificulte la regresión a las normas de antes y que facilite medidas de libertad.}

% ============================================================================
\section{Resumen Ejecutivo: Las Cuatro Fases}
% ============================================================================

\begin{center}
\begin{longtable}{|p{0.15\textwidth}|p{0.25\textwidth}|p{0.5\textwidth}|}
\hline
\textbf{Fase} & \textbf{Objetivo} & \textbf{Medidas Clave} \\
\hline
\endfirsthead
\hline
\textbf{Fase} & \textbf{Objetivo} & \textbf{Medidas Clave} \\
\hline
\endhead

{\color{fase1}\textbf{I}} & Aprovechar marco existente &
Competencia fiscal autonómica, extensión modelo foral, arbitraje comercial, mediación comunitaria \\
\hline

{\color{fase2}\textbf{II}} & Descentralizar y fragmentar &
Transferencia a municipios, comunidades vecinales autogestionadas, desregulación sectorial \\
\hline

{\color{fase3}\textbf{III}} & Privatización real &
Cheque escolar, seguros sanitarios privados, pensiones de capitalización, seguridad privada \\
\hline

{\color{fase4}\textbf{IV}} & Pluralismo jurídico &
Derecho consuetudinario, tribunales privados, Lex Mercatoria, secesión municipal \\
\hline

\end{longtable}
\end{center}

% ============================================================================
\section{Conclusión: El Camino de la Libertad}
% ============================================================================

Este plan no pretende ser un programa de gobierno ni una hoja de ruta con fechas concretas. Siguiendo el espíritu de Bastos, las mejores transiciones son aquellas que ocurren orgánicamente porque las nuevas instituciones demuestran ser superiores a las antiguas.

El objetivo es crear las \textbf{condiciones de posibilidad} para que emerja un orden social basado en:

\begin{itemize}
    \item La \textbf{propiedad privada} como fundamento de todas las libertades.
    \item El \textbf{derecho natural} como fuente de legitimidad jurídica.
    \item La \textbf{comunidad} como unidad básica de organización social.
    \item El \textbf{contrato} como forma de relación entre personas y grupos.
    \item La \textbf{competencia institucional} como mecanismo de mejora continua.
\end{itemize}

\cita{El anarcocapitalismo consiste en establecer una sociedad basada en los principios de libre mercado en ausencia de Estado.}

España, con su rica tradición de fueros, concejos, gremios y comunidades locales, posee un patrimonio institucional que puede servir de inspiración para un futuro de libertad. No se trata de importar modelos foráneos, sino de recuperar lo mejor de nuestra propia historia.

\vspace{1cm}

\begin{center}
\fbox{\parbox{0.9\textwidth}{
    \centering
    \textbf{Principio Final}\\[0.3cm]
    \textit{``Rothbard decía que la aportación principal de la teoría política hispana de todos los tiempos es el concepto de `fuero'. Porque es un concepto que frena el poder del Estado.''}\\[0.2cm]
    --- Miguel Anxo Bastos
}}
\end{center}

\end{document}
