\documentclass[11pt,a4paper]{article}
\usepackage[utf8]{inputenc}
\usepackage[spanish]{babel}
\usepackage[margin=2.5cm]{geometry}
\usepackage{enumitem}
\usepackage{titlesec}
\usepackage{xcolor}
\usepackage{hyperref}
\usepackage{needspace}

\definecolor{primary}{RGB}{45,55,72}
\hypersetup{colorlinks=true,linkcolor=primary,urlcolor=primary}

\titleformat{\section}{\large\bfseries\color{primary}}{\thesection.}{0.5em}{}
\titlespacing*{\section}{0pt}{1.5ex}{1ex}

\setlength{\parindent}{0pt}
\setlength{\parskip}{0.8em}

% Evitar viudas y huérfanas
\widowpenalty=10000
\clubpenalty=10000

\pagestyle{empty}

\begin{document}

\begin{center}
{\LARGE\bfseries PayloadAgents}\\[0.3em]
{\large Framework de Gestión de Contenido para Agentes de IA}\\[0.5em]
{\small Nexo Labs --- \url{https://github.com/Nexo-Labs/PayloadAgents}}
\end{center}

\vspace{0.5em}

\needspace{5\baselineskip}
\section*{Descripción}

PayloadAgents es un framework multi-tenant para la administración de contenido orientado a agentes de IA. Construido sobre Payload CMS, proporciona la infraestructura de datos necesaria para alimentar sistemas de recuperación aumentada (RAG) y agentes conversacionales.

El objetivo del framework es estructurar, indexar y exponer contenido de forma óptima para su consumo por sistemas de IA, independientemente del motor RAG o base de datos vectorial que se utilice como backend.

\needspace{8\baselineskip}
\section*{Contexto de Aplicación}

\begin{samepage}
Este framework se está desarrollando como base tecnológica para la creación de una \textbf{plataforma de análisis del conocimiento asistida por inteligencia artificial}. 

En concreto, el caso de uso principal es la construcción de una plataforma para el estudio de autores y materias de la \textbf{Escuela Austriaca de Economía}, permitiendo explorar, relacionar y consultar de forma inteligente el corpus de obras y pensamiento de esta tradición intelectual.
\end{samepage}

\needspace{12\baselineskip}
\section*{Objetivos del Proyecto}

\begin{samepage}
\begin{itemize}[leftmargin=1.5em, itemsep=0.3em]
    \item \textbf{Gestión de contenido multi-tenant:} Aislamiento completo de datos por organización
    \item \textbf{Indexación inteligente:} Chunking, embeddings y sincronización con motores de búsqueda
    \item \textbf{Arquitectura extensible:} Diseño modular que permite integrar diferentes backends de RAG
    \item \textbf{Taxonomías automáticas:} Clasificación de contenido asistida por IA
    \item \textbf{Autenticación empresarial:} Integración con Keycloak para SSO
\end{itemize}
\end{samepage}

\needspace{10\baselineskip}
\section*{Estado Actual}

\begin{samepage}
Actualmente, el framework ofrece integración nativa con \textbf{Typesense} como motor de búsqueda vectorial y RAG básico. Esta implementación permite búsqueda híbrida (semántica + keyword) y respuestas conversacionales sobre el contenido indexado.

La arquitectura está diseñada para ser extensible: nada impide integrar otras bases de datos vectoriales (Pinecone, Weaviate, Qdrant) u otros sistemas RAG personalizados según las necesidades del proyecto.

\textit{Nota: En un proyecto paralelo se está explorando la integración de Self-RAG con la arquitectura de datos que genera PayloadAgents.}
\end{samepage}

\needspace{10\baselineskip}
\section*{Componentes}

\begin{samepage}
El monorepo incluye cinco paquetes npm bajo el scope \texttt{@nexo-labs}:

\begin{itemize}[leftmargin=1.5em, itemsep=0.2em]
    \item \texttt{payload-typesense} --- Sincronización e indexación con Typesense
    \item \texttt{payload-indexer} --- Extracción, chunking y generación de embeddings
    \item \texttt{payload-taxonomies} --- Clasificación automática con IA
    \item \texttt{payload-stripe-inventory} --- Integración con Stripe
    \item \texttt{chat-agent} --- Componente React de interfaz de chat
\end{itemize}
\end{samepage}

\needspace{8\baselineskip}
\section*{Stack Tecnológico}

\begin{samepage}
\begin{tabular}{@{}ll@{}}
\textbf{CMS:} & Payload CMS 3.x (Next.js 15, React 19) \\
\textbf{Base de datos:} & MongoDB / PostgreSQL \\
\textbf{Búsqueda:} & Typesense (extensible a otros backends) \\
\textbf{Embeddings:} & OpenAI, Google Gemini \\
\textbf{Auth:} & Keycloak, NextAuth \\
\end{tabular}
\end{samepage}

\vfill

\begin{center}
{\small\color{gray} MIT License --- Versión 1.8.0}
\end{center}

\end{document}
